\documentclass[a4paper]{article}

\usepackage[english]{babel}
\usepackage[utf8]{inputenc}
\usepackage{amsmath}
\usepackage{graphicx}
%\usepackage[colorinlistoftodos]{todonotes}

\title{Elaborazione di dati tridimensionali: homework 2}

\author{Alberto Cenzato 1134707}

\date{\today}

\begin{document}
\maketitle

\begin{abstract}
Scopo di questo secondo homework era di familiarizzare con la libreria PCL. Sono stati svolti quattro esercizi di complessità crescente per approfondire diverse funzionalità di PCL: lettura di point cloud da file, lettura e visualizzazione dei valori contenuti nella point cloud, downsampling, calcolo di normali, keypoint e descrittori, registrazione di point cloud, people detection.
\end{abstract}

\section{Lab 1} \label{sec:lab1}

Explain the context of the experiment here. Why is condensed matter physics interesting or important?
Optional things you could talk about (but don't have to -- this is up to you): transistors, computers, Quantum computers, fundamental knowledge (e.g. the resistance quantum).

Briefly explain what methods you will use in the experiment, and what values you will extract from the data.

For this section and all following sections: If you refer to an equation, previous result or theory that is not regarded as common knowledge, then cite the source (article or book) where you found this. For example, you can cite the Nano 3 Lecture notes \cite{nano3}.

\section{Lab 2} \label{sec:lab2}

\subsection{Two-dimensional Electron Gas}
Here, explain the concept of a 2-DEG in GaAs/AlGaAs. What is a 2-DEG and why does it arise?

\subsection{Hall Effect}
Explain the classical Hall effect in your own words. What do I measure at $B=0$? And what happens if $B>0$? Which effect gives rise to the voltage drop in the vertical direction?

\subsection{Quantum Hall Effect}
Explain the IQHE in your own words. What does the density of states look like in a 2-DEG when $B=0$? What are Landau levels and how do they arise? What are edge states? What does the electron transport look like when you change the magnetic field? What do you expect to measure?

\section{Lab 3} \label{sec:lab3}
\subsection{Fabrication}
Explain a step-by-step recipe for fabrication here. How long did you etch and why? What is an Ohmic contact?
\subsection{Experimental set-up}
Explain the experimental set-up here. Use a schematic picture (make it yourself in photoshop, paint, ...) to show how the components are connected. Briefly explain how a lock-in amplifier works.

\section{Lab 4} \label{sec:lab4}
Show a graph of the longitudinal resistivity ($\rho_{xx}$) and Hall resistivity ($\rho_{xy}$) versus magnetic field, extracted from the raw data shown in figure \ref{fig:data}. You will have the link to the data in your absalon messages, if not e-mail Guen (guen@nbi.dk). Explain how you calculated these values, and refer to the theory.

\begin{figure}
\centering
\includegraphics[width=1\textwidth]{images/raw_data.jpg}
\caption{\label{fig:data}Raw (unprocessed) data. Replace this figure with the one you've made, that shows the resistivity.}
\end{figure}

\subsection{Classical regime}
Calculate the sheet electron density $n_{s}$ and electron mobility $\mu$ from the data in the low-field regime, and refer to the theory in section \ref{sec:theory}. Explain how you retrieved the values from the data (did you use a linear fit?).
Round values off to 1 or 2 significant digits: 8.1643 ~= 8.2. Also, 5e-6 is easier to read than 0.000005.

!OBS: This part is optional (only if you have time left).
Calculate the uncertainty as follows: \newline $u(f(x, y, z)) = \sqrt{(\frac{\delta f}{\delta{x}} u(x))^{2} + (\frac{\delta f}{\delta{y}} u(y))^{2} + (\frac{\delta f}{\delta{z}} u(z))^{2}}$, where $f$ is the calculated value ($n_{s}$ or $\mu$), $x, y, z$ are the variables taken from the measurement and $u(x)$ is the uncertainty in x (and so on).

\subsection{Quantum regime}
Calculate $n_{s}$ for the high-field regime.
Show a graph of the longitudinal conductivity ($\rho_{xx}$) and Hall conductivity($\rho_{xy}$) \textbf{in units of the resistance quantum} ($\frac{h}{e^{2}}$), depicting the integer filling factors for each plateau.
Show a graph of the plateau number versus its corresponding value of $1/B$. From this you can determine the slope, which you use to calculate the electron density.
Again, calculate the uncertainty for your obtained values.

\begin{thebibliography}{9}
\bibitem{nano3}
  K. Grove-Rasmussen og Jesper Nygård,
  \emph{Kvantefænomener i Nanosystemer}.
  Niels Bohr Institute \& Nano-Science Center, Københavns Universitet

\end{thebibliography}
\end{document}